\documentclass[11pt,]{article}
\usepackage[]{geometry}
\newcommand*{\authorfont}{\fontfamily{phv}\selectfont}
\usepackage[]{mathpazo}
\usepackage{abstract}
\renewcommand{\abstractname}{}    % clear the title
\renewcommand{\absnamepos}{empty} % originally center
\newcommand{\blankline}{\quad\pagebreak[2]}

\providecommand{\tightlist}{%
  \setlength{\itemsep}{0pt}\setlength{\parskip}{0pt}} 
\usepackage{longtable,booktabs}

\usepackage{parskip}
\usepackage{titlesec}
\titlespacing\section{0pt}{12pt plus 4pt minus 2pt}{6pt plus 2pt minus 2pt}
\titlespacing\subsection{0pt}{12pt plus 4pt minus 2pt}{6pt plus 2pt minus 2pt}

\usepackage{titling}
\setlength{\droptitle}{-.25cm}

%\setlength{\parindent}{0pt}
%\setlength{\parskip}{6pt plus 2pt minus 1pt}
%\setlength{\emergencystretch}{3em}  % prevent overfull lines 

\usepackage[T1]{fontenc}
\usepackage[utf8]{inputenc}

\usepackage{fancyhdr}
\pagestyle{fancy}

\usepackage{lastpage}
\renewcommand{\headrulewidth}{0.3pt}
\renewcommand{\footrulewidth}{0.0pt} 
\lhead{}
\chead{}
\rhead{\footnotesize POLI 618: Techniques of Empirical Research -- Autumn 2017}
\lfoot{}
\cfoot{\small \thepage/\pageref*{LastPage}}
\rfoot{}

\fancypagestyle{firststyle}
{
\renewcommand{\headrulewidth}{0pt}%
   \fancyhf{}
   \fancyfoot[C]{\small \thepage/\pageref*{LastPage}}

}

%\def\labelitemi{--}
%\usepackage{enumitem}
%\setitemize[0]{leftmargin=25pt}
%\setenumerate[0]{leftmargin=25pt}




\makeatletter
\@ifpackageloaded{hyperref}{}{%
\ifxetex
  \usepackage[setpagesize=false, % page size defined by xetex
              unicode=false, % unicode breaks when used with xetex
              xetex]{hyperref}
\else
  \usepackage[unicode=true]{hyperref}
\fi
}
\@ifpackageloaded{color}{
    \PassOptionsToPackage{usenames,dvipsnames}{color}
}{%
    \usepackage[usenames,dvipsnames]{color}
}
\makeatother
\hypersetup{breaklinks=true,
            bookmarks=true,
            pdfauthor={ ()},
             pdfkeywords = {},  
            pdftitle={POLI 618: Techniques of Empirical Research},
            colorlinks=true,
            citecolor=blue,
            urlcolor=blue,
            linkcolor=magenta,
            pdfborder={0 0 0}}
\urlstyle{same}  % don't use monospace font for urls


\setcounter{secnumdepth}{3} %add section headers





\usepackage{setspace}

\title{POLI 618: Techniques of Empirical Research}
\author{Professor Aaron Erlich}
\date{Autumn 2017}


\begin{document}  

		\maketitle
			\thispagestyle{firststyle}

%	\thispagestyle{empty}


\noindent \begin{tabular*}{\textwidth}{ @{\extracolsep{\fill}} lr @{\extracolsep{\fill}}}
E-mail: \texttt{\href{mailto:aaron.erlich@mcgill.ca}{\nolinkurl{aaron.erlich@mcgill.ca}}} & Phone: {\texttt 514-398-4756}\\
Office Hours: W 10-11 \href{http://www.aaronerlich.com/office-hours}{(by appt)}, F
10-11 (drop-in)  &  Class Hours: W,F 8:35 AM - 9:55 AM\\
Office: 3610 McTavish, Room 26-4  & Class Room: MUSIC A-412\\
	&  \\ 
	\hline
	\end{tabular*}
  
\noindent \begin{tabular*}{\textwidth}{ @{\extracolsep{\fill}} lr @{\extracolsep{\fill}}}

TA: \textbf{Aengus Bridgman} & E-mail: \texttt{\href{mailto:aengus.bridgman@mail.mcgill.ca}{\nolinkurl{aengus.bridgman@mail.mcgill.ca}}} \\
Office Hours: TBD  &
                                                                   Conference
                                                                       Hours:
                                                                       TBD\\
Office: TBD  &  Conference Room: Check
                                                         your schedule \\ 
	\hline
\end{tabular*}	
\vspace{10mm}

\section{Course Description}\label{course-description}

This course is designed to introduce graduate students to the new and
exciting world of data driven political analysis. The course employs
examples from across political science sub-disciplines and is generally
relevant to all social science research. The course is divided into
lectures and labs.

\subsection{Who this course is for?}\label{who-this-course-is-for}

\begin{itemize}
\item
  This is your first semester-long course on graduate quantitative
  methods course for data analysis.
\item
  You are willing to spend time considerable outside of the classroom to
  learn the course materials, as data analysis is a skill learned by
  doing.
\end{itemize}

\section{Course Objectives}\label{course-objectives}

\begin{enumerate}
\def\labelenumi{\arabic{enumi}.}
\item
  Learn the basic tools of empirical research in political science.
\item
  Obtain skills in R, a highly powerful and FREE programming language
  used extensively by academics in political science across the world,
  as well as the open source and data science community.
\item
  Understand scientific replicability
\item
  Gain real world skills that will help you obtain jobs in careers of
  the future.
\end{enumerate}

\section{Textbooks}\label{textbooks}

Given people's various backgrounds, we will have three required and
several optional textbooks in addition to the assigned articles. The
Bailey book is a great book and very applied. Everyone should read it.
For those of you seriously interested in pursuing quantitative analysis,
you should then read the Fox book on top of the Bailey book. Unless
otherwise cleared with me, everyone is required to do the math review
problems assigned from the Moore book. The Grolemund and Wickham book is
a great tool (that is online and free) and can often be used in lieu of
videos or to help with coding.

\inputencoding{utf8} Bailey, Michael A. (2015).
\emph{Real Stats: Using Econometrics for Political
Science and Public Policy}. 1st. New York: Oxford University Press.

\inputencoding{utf8} Moore, Will H. and David A. Siegel (2013).
\emph{A Mathematics Course for Political and Social Research}. Princeton
University Press.

\inputencoding{utf8} Grolemund, Garrett and Hadley Wickham (2016).
\emph{R for Data Science}. \url{http://r4ds.had.co.nz/}.

\section{Recommended}\label{recommended}

\inputencoding{utf8} Fox, John (2015).
\emph{Applied Regression Analysis and Generalized
Linear Models}. 3rd ed.. Los Angeles: SAGE Publications, Inc. ISBN:
978-1-4522-0566-3.

\section{R and R-Studio}\label{r-and-r-studio}

R is a FREE open source programming language uses by data scientists and
statisticians across the world. We will be using a FREE graphical user
interface (GUI) called \textbf{RStudio} that makes learning R easier.
While the learning curve in R is steeper than expensive programs, there
is much more you can do with it!

There are many free online tutorial for downloading and installing R and
RStudio. R will also be available on the conference computers.

\section{Evaluation Policy}\label{evaluation-policy}

A description of the means of evaluation to be used in the course:

There are 1000 points available in the class. Therefore, for each 10\%
of the grade, 100 points are allotted.

\begin{itemize}
\item
  \textbf{10\%}. \emph{Attendance and participation.}
\item
  \textbf{40\%}. \emph{Problem sets}. A quarter of this grade will be
  from peer-reviewing each others' assignments on \emph{GitHub}.
\item
  \textbf{10\%} \emph{Final Presentation}. In class presentation of
  research.
\item
  \textbf{40\%} \emph{Final Paper}. A paper --- likely a replication
  paper or part of your dissertation or MA Thesis.
\end{itemize}

\subsection{Re-Grading}\label{re-grading}

Students who wish to contest a grade for an assignment or exam must do
so in writing (by email, sent to me) providing the reasoning behind
their challenge to the grade received within two weeks of the day on
which the assignments are returned. The TA who graded the assignment
will re-grade your assignment, and may \textbf{raise or lower the
grade}. If you are still unsatisfied after the re-assessment, you can
re-submit the assignment to me (original copy with TA comments), along
with your justification. I will then re-evaluate the paper, but also
reserve the right to \textbf{raise or lower the grade}. Please also see
(\url{http://www.mcgill.ca/politicalscience/files/politicalscience/assessment_and_re-read_policy_final.pdf}).

\section{Assignment Submission}\label{assignment-submission}

Problem sets must be submitted via \emph{GitHub}. You must use a variant
\texttt{.Rmd} otherwise known as \texttt{rmarkdown} files that is GitHub
to complete your homework. Do not submit your homework using Microsft
Word or any other document editor. It will not be graded. Part of your
homework grade will be peer-reviewing your colleagues' homework
assignments via the course \emph{GitHub}. There will be approximately 5
problem sets. I reserve the right to lower the number of problem sets if
needed. In principle, these assignments will be due every 2 weeks
starting week 2 for the proceeding 12 weeks of the term. Collaboration
is part of learning how to code. I encourage you to collaborate! But you
do not learn how to do statistical programming if you DO NOT write your
own code. Please feel free to collaborate with colleagues, but please DO
NOT copy each others' code verbatim. You must also write your own
interpretations of the results.

\subsection{Interim Data Set and
Check-in}\label{interim-data-set-and-check-in}

All students must submit a one page write up of the data set they are
going to use and the research question they are going to ask by Tuesday,
Feb 16. This should be a one page write up in \texttt{rmarkdown}
explaining the data set/s which you are going to use and the question
you will ask. You should also highlight your outcome variable.

\subsection{Presentation}\label{presentation}

Each individual will present their work in class to the rest of the
course. All presentations must be completed in .Rpres,
revealjs\_presentation, or Beamer (LaTeX).

\subsection{Paper}\label{paper}

All students will submit a final paper that is of article length. This
will be done via \emph{MyCourses}. This can be \emph{either} a
replication paper with an extension of the original paper \emph{or} a
new paper. It is highly recommend you do a replication. Please come talk
me as early in the term as possible if you want to write an original
paper. An original paper must contain 1) a clear theory proposing a
relationship between explanatory variable(s) on an outcome variable; 2)
use of linear regression (or some other model cleared with professor);
3) a clear discussion of both findings and limitations of the paper.
Students may use a chapter of their master or Ph.D.~thesis as a research
paper. The paper is due by the end of the exam period Wednesday, Dec 21.

\section{Class Discussion List and E-mail
Policy}\label{class-discussion-list-and-e-mail-policy}

I have set up a class discussion list on \emph{MyCourses}. I encourage
you to use this mailing list to ask questions you may have. \emph{NEVER}
post your code or specific homework questions on the course list server.
Please post general questions! If you post homework code on the website,
it will be taken down and your grade may be lowered.

\section{Make-Up Work Policy}\label{make-up-work-policy}

If you are absent for documented emergency medical or family reasons, an
alternative homework submission date will be arranged. The alternative
arrangement is only open to those who can provide a valid medical/family
reason for missing the quizzes. If you cannot provide a valid reason for
your absence, you will receive 0 points for the missed quiz or homework
submission.

Students who need to miss a class due to a religious holiday should
notify me at least fourteen days prior to the holiday. If you must miss
a class, an examination, a work assignment, or a project in order to
observe a religious holy day, you will be given an opportunity to
complete the missed work within a reasonable time after the absence.

\section{Technology Policy}\label{technology-policy}

\subsection{Recording Policy}\label{recording-policy}

\begin{itemize}
\tightlist
\item
  No audio or video recording of any kind is allowed in class without
  the explicit permission of the instructor.
\item
  Mobile Computing devices are not to be used for voice communication
  without the explicit permission of the instructor.
\end{itemize}

\section{Academic Integrity}\label{academic-integrity}

\subsection{Course Policy on Computer
Code}\label{course-policy-on-computer-code}

As discussed in the problems set section, verbatim copying other
people's computer code constitutes plagiarism. Moreover, data
programming is learned through trial and error. \textbf{Please do not
under any circumstances copy another students code.} You may of course
collaborate with colleagues, but please write your own code! If you are
found to have plagiarized, you may be referred to the appropriate Dean.
The instructors reserve the right to use software to compare the code
that has been written by different students.

\subsection{McGill Policy}\label{mcgill-policy}

``McGill University values academic integrity. Therefore, all students
must understand the meaning and consequences of cheating, plagiarism and
other academic offences under the Code of Student Conduct and
Disciplinary Procedures'' (see \url{www.mcgill.ca/students/srr/honest/}
for more information).

\section{Other Policies}\label{other-policies}

\subsection{Language of Submission:}\label{language-of-submission}

In accord with McGill University's Charter of Students' Rights, students
in this course have the right to submit in English or in French any
written work that is to be graded.

\subsection{Disabilities Policy}\label{disabilities-policy}

As the instructor of this course I endeavor to provide an inclusive
learning environment. However, if you experience barriers to learning in
this course, do not hesitate to discuss them with me and the Office for
Students with Disabilities, 514-398-6009.

\subsection{End of Course Evaluations}\label{end-of-course-evaluations}

End-of-course evaluations are one of the ways that McGill works towards
maintaining and improving the quality of courses and the student's
learning experience. You will be notified by e-mail when the evaluations
are available. Please note that a minimum number of responses must be
received for results to be available to students. \newpage

\section{Class Schedule}\label{class-schedule}

\paragraph{Week 01, 09/04 - 09/08: Introduction, Intro to
R}\label{week-01-0904---0908-introduction-intro-to-r}

\begin{itemize}
\tightlist
\item
  VIDEOS:

  \begin{itemize}
  \tightlist
  \item
    Data camp:
    \href{https://www.datacamp.com/courses/free-introduction-to-r-beta}{Intro
    R lectures} Chapters 1 (Intro to Basics), 2 (Vectors), 4 (Factors),
    and 5 (Data Frames)
  \item
    Data camp:
    \href{https://www.datacamp.com/courses/reporting-with-r-markdown}{R-markdown
    lectures}
  \end{itemize}
\item
  READING:

  \begin{itemize}
  \tightlist
  \item
    Moore Ch. 1
  \item
    Bailey Ch. 1
  \end{itemize}
\item
  TASKS

  \begin{itemize}
  \tightlist
  \item
    Complete \href{https://try.GitHub.io/levels/1/challenges/1}{tryGit}
  \item
    Install git on your computer and push a test file to GitHub
  \end{itemize}
\end{itemize}

\paragraph{Week 02, 09/11 - 09/15: Introduction to R-Markdown, Managing
Data, Replication, Git and
GitHub}\label{week-02-0911---0915-introduction-to-r-markdown-managing-data-replication-git-and-github}

\begin{itemize}
\tightlist
\item
  VIDEOS:

  \begin{itemize}
  \tightlist
  \item
    Data camp:
    \href{https://www.datacamp.com/courses/dplyr-data-manipulation-r-tutorial}{Data
    minipulation lectures}
  \end{itemize}
\item
  READING:

  \begin{itemize}
  \item
    Moore Chapter 5, 6
  \item
    Bailey Chapter 2
  \item
    Grolemund \& Wikham Ch. 27
  \item
    \inputencoding{utf8} Dafoe, Allan (2014). ``Science Deserves Better:
    The Imperative to Share Complete Replication Files''. In:
    \emph{PS: Political Science \& Politics} 47.1, pp.~60--66.
  \item
    \inputencoding{utf8} Eubank, Nicholas (2016). ``Embrace your
    Fallibility: Thoughts on Code Integrity''. In:
    \emph{The Political Methodologist} 23.2, pp.~10--15. (Visited on
    Nov. 21, 2016).
  \item
    \inputencoding{utf8} Moravcsik, Andrew (2010). ``Active Citation: A
    Precondition for Replicable Qualitative Research''. In:
    \emph{PS: Political Science \& Politics} 43.01, pp.~29--35. (Visited
    on Nov. 30, 2016).
  \end{itemize}
\end{itemize}

\paragraph{Week 03, 09/18 - 09/22: Probability and Statistical
Inference}\label{week-03-0918---0922-probability-and-statistical-inference}

\begin{itemize}
\tightlist
\item
  READING:

  \begin{itemize}
  \tightlist
  \item
    Moore Chapter 9
  \item
    Galimard Chapter 3 (photocopy)
  \end{itemize}
\end{itemize}

\paragraph{Week 04, 09/25 - 09/29: Intro to graphics, Linear Regression
Basics
Review}\label{week-04-0925---0929-intro-to-graphics-linear-regression-basics-review}

\begin{itemize}
\tightlist
\item
  VIDEOS

  \begin{itemize}
  \tightlist
  \item
    Data camp:
    \href{https://www.datacamp.com/courses/data-visualization-with-ggplot2-1}{ggplot
    lectures}
  \end{itemize}
\item
  READING:

  \begin{itemize}
  \item
    Moore 2 (if you need algebra review)
  \item
    Bailey Chapters 3, 4, 5
  \item
    Fox Chapter 2
  \item
    Grolemund Chapter 3
  \item
    \inputencoding{utf8} Kastellec, Jonathan P. and Eduardo L. Leoni
    (2007). ``Using Graphs Instead of Tables in Political Science''. In:
    \emph{Perspectives on Politics} 5.4, pp.~755--771. ISSN: 1541-0986,
    1537-5927. (Visited on Oct. 25, 2016).
  \item
    \inputencoding{utf8} King, Gary, Michael Tomz and Jason Wittenberg
    (2000). ``Making the Most of Statistical Analyses: Improving
    Interpretation and Presentation''. In:
    \emph{American Journal of Political Science} 44.2, pp.~347--361.
    ISSN: 0092-5853. (Visited on Apr. 28, 2011).
  \end{itemize}
\end{itemize}

\paragraph{Week 05, 10/02 - 10/06: Data
Transformation}\label{week-05-1002---1006-data-transformation}

\begin{itemize}
\tightlist
\item
  READING:

  \begin{itemize}
  \tightlist
  \item
    Moore Chapter 3
  \item
    Bailey Chapters 6.1-6.3, 7
  \item
    Fox Chapters 4, 7.1-7.2
  \end{itemize}
\end{itemize}

\paragraph{Week 06, 10/09 - 10/13: Interaction Terms,
Simulation}\label{week-06-1009---1013-interaction-terms-simulation}

\begin{itemize}
\tightlist
\item
  READING:

  \begin{itemize}
  \item
    Moore Chapter 7
  \item
    Bailey 6.4
  \item
    Fox 7.3
  \item
    \inputencoding{utf8} Berry, William D, Matt Golder and Daniel Milton
    (2012). ``Improving Tests of Theories Positing Interaction''. En.
    In: \emph{The Journal of Politics} 74.3, pp.~653--671. ISSN:
    0022-3816, 1468-2508.
  \item
    \inputencoding{utf8} Brambor, Thomas, William Roberts Clark and Matt
    Golder (2006). ``Understanding Interaction Models: Improving
    Empirical Analyses''. En. In: \emph{Political Analysis}, pp.~63--82.
  \end{itemize}
\item
  \textbf{PAPER CHECK IN : Tuesday, Feb 16 }
\end{itemize}

\paragraph{Week 07, 10/16 - 10/20: Assumptions \& Properties of the
Linear
Regression}\label{week-07-1016---1020-assumptions-properties-of-the-linear-regression}

\begin{itemize}
\tightlist
\item
  READING:

  \begin{itemize}
  \tightlist
  \item
    Bailey Ch. 14
  \item
    Fox Ch.6
  \end{itemize}
\end{itemize}

\paragraph{Week 08, 10/23 - 10/27: Matrix Presentation of
LS}\label{week-08-1023---1027-matrix-presentation-of-ls}

\begin{itemize}
\tightlist
\item
  READING:

  \begin{itemize}
  \tightlist
  \item
    Moore Chapter 12
  \item
    Fox Chapter 9.1-9.2
  \end{itemize}
\end{itemize}

\paragraph{Week 09, 10/30 - 11/03 : LS and Causal
Inference}\label{week-09-1030---1103-ls-and-causal-inference}

\begin{itemize}
\tightlist
\item
  READING:

  \begin{itemize}
  \tightlist
  \item
    TBD
  \item
    TBD
  \end{itemize}
\item
  NO CLASS NOVEMBER 31, WORK ON YOUR PAPERS
\end{itemize}

\paragraph{Week 10, 11/06 - 11/10: Linear Regression Diagnostics and
Fixes}\label{week-10-1106---1110-linear-regression-diagnostics-and-fixes}

\begin{itemize}
\tightlist
\item
  READING:

  \begin{itemize}
  \tightlist
  \item
    Moore Chapter 12
  \item
    Fox Chapter 9.1-9.2
  \end{itemize}
\end{itemize}

\paragraph{Week 11, 11/13 - 11/17: Logit/Probit and Linear Probability
Model}\label{week-11-1113---1117-logitprobit-and-linear-probability-model}

\begin{itemize}
\tightlist
\item
  READING:

  \begin{itemize}
  \tightlist
  \item
    Bailey Ch. 12
  \item
    Fox Ch. 14
  \end{itemize}
\end{itemize}

\paragraph{Week 12, 11/20 - 11/24: Presentations
1}\label{week-12-1120---1124-presentations-1}

\paragraph{Week 13, 11/27 - 12/01: Presentations
2}\label{week-13-1127---1201-presentations-2}

\section{Recommended Textbooks}\label{recommended-textbooks}

There are many other important textbooks and at some point you may find
yourself looking for a different explanation of something you didn't
understand -- or looking to go deeper. Here are some places to start.

\inputencoding{utf8} Angrist, Joshua D. and Jörn-Steffen Pischke (2008).
\emph{Mostly Harmless Econometrics: An Empiricist's Companion}.
Princeton University Press. ISBN: 978-1-4008-2982-8.

\inputencoding{utf8} Gailmard, Sean (2014).
\emph{Statistical Modeling and Inference for Social Science}. New York,
NY: Cambridge University Press.

\inputencoding{utf8} Gelman, Andrew and Jennifer Hill (2007).
\emph{Data Analysis Using Regression and
Multilevel/Hierarchical Models}. New York: Cambridge University Press.

\inputencoding{utf8} Greene, William H. (2003).
\emph{Econometric analysis}. Pearson Education.




\end{document}

\makeatletter
\def\@maketitle{%
  \newpage
%  \null
%  \vskip 2em%
%  \begin{center}%
  \let \footnote \thanks
    {\fontsize{18}{20}\selectfont\raggedright  \setlength{\parindent}{0pt} \@title \par}%
}
%\fi
\makeatother
