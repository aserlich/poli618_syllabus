\documentclass[11pt,]{article}
\usepackage[margin=1in]{geometry}
\newcommand*{\authorfont}{\fontfamily{phv}\selectfont}
\usepackage[]{mathpazo}
\usepackage{abstract}
\renewcommand{\abstractname}{}    % clear the title
\renewcommand{\absnamepos}{empty} % originally center
\newcommand{\blankline}{\quad\pagebreak[2]}

\providecommand{\tightlist}{%
  \setlength{\itemsep}{0pt}\setlength{\parskip}{0pt}} 
\usepackage{longtable,booktabs}

\usepackage{parskip}
\usepackage{titlesec}
\titlespacing\section{0pt}{12pt plus 4pt minus 2pt}{6pt plus 2pt minus 2pt}
\titlespacing\subsection{0pt}{12pt plus 4pt minus 2pt}{6pt plus 2pt minus 2pt}

\usepackage{titling}
\setlength{\droptitle}{-.25cm}

%\setlength{\parindent}{0pt}
%\setlength{\parskip}{6pt plus 2pt minus 1pt}
%\setlength{\emergencystretch}{3em}  % prevent overfull lines 

\usepackage[T1]{fontenc}
\usepackage[utf8]{inputenc}

\usepackage{fancyhdr}
\pagestyle{fancy}

\usepackage{lastpage}
\renewcommand{\headrulewidth}{0.3pt}
\renewcommand{\footrulewidth}{0.0pt} 
\lhead{}
\chead{}
\rhead{\footnotesize POLS 311: Techniques of Empirical Research -- Fall 2016}
\lfoot{}
\cfoot{\small \thepage/\pageref*{LastPage}}
\rfoot{}

\fancypagestyle{firststyle}
{
\renewcommand{\headrulewidth}{0pt}%
   \fancyhf{}
   \fancyfoot[C]{\small \thepage/\pageref*{LastPage}}

}

%\def\labelitemi{--}
%\usepackage{enumitem}
%\setitemize[0]{leftmargin=25pt}
%\setenumerate[0]{leftmargin=25pt}




\makeatletter
\@ifpackageloaded{hyperref}{}{%
\ifxetex
  \usepackage[setpagesize=false, % page size defined by xetex
              unicode=false, % unicode breaks when used with xetex
              xetex]{hyperref}
\else
  \usepackage[unicode=true]{hyperref}
\fi
}
\@ifpackageloaded{color}{
    \PassOptionsToPackage{usenames,dvipsnames}{color}
}{%
    \usepackage[usenames,dvipsnames]{color}
}
\makeatother
\hypersetup{breaklinks=true,
            bookmarks=true,
            pdfauthor={ ()},
             pdfkeywords = {},  
            pdftitle={POLS 311: Techniques of Empirical Research},
            colorlinks=true,
            citecolor=blue,
            urlcolor=blue,
            linkcolor=magenta,
            pdfborder={0 0 0}}
\urlstyle{same}  % don't use monospace font for urls


\setcounter{secnumdepth}{3} %add section headers





\usepackage{setspace}

\title{POLS 311: Techniques of Empirical Research}
\author{Aaron Erlich}
\date{Fall 2016}


\begin{document}  

		\maketitle
		
	
		\thispagestyle{firststyle}

%	\thispagestyle{empty}


	\noindent \begin{tabular*}{\textwidth}{ @{\extracolsep{\fill}} lr @{\extracolsep{\fill}}}


E-mail: \texttt{\href{mailto:aaron.erlich@mcgill.ca}{\nolinkurl{aaron.erlich@mcgill.ca}}} & Phone: {\texttt 514-398-4400 ext. 3342 (until Sep 24)}\\
Office Hours: T 14-15 \href{http://www.aaronerlich.com/office-hours}{(by appt)}, Th
14-15 (drop-in)  &  Class Hours: T-Th 16:05-17:25\\
Office: 24-3 3610 McTavish (until Sep 24)  & Class Room: Burnside Hall 1B45\\
	&  \\ 
	\hline
	\end{tabular*}

	\noindent \begin{tabular*}{\textwidth}{ @{\extracolsep{\fill}} lr @{\extracolsep{\fill}}}
TA \textbf{Chris Chhim} & TA E-mail: \texttt{\href{mailto:chris.chhim@mail.mcgill.ca}{\nolinkurl{chris.chhim@mail.mcgill.ca}}} \\
TA Office Hours: TBD  &
                                                                   Conference
                                                                   Hours:
                                                                   Check
                                                                   your schedule
                                                                   \\
TA Office: TBD  &  Conference
                                                       Room: Leacock
                                                       212 \\ 
	\hline
\end{tabular*}
	
\vspace{2mm}
	


\section{Course Description}\label{course-description}

This course is designed to introduce students to the new and exciting
world of data driven political analysis. The course employs examples
from across political science sub-disciplines to discuss questions such
as whether government spending helps reduce poverty or whether contact
with canvassers can change voters' minds The course also provides
students real world skills that they can put on the resumés when
applying for jobs. Data science is one of the fastest and most important
industries out there! And political scientists are playing a major role
in the field.

\subsection{Who this course is for?}\label{who-this-course-is-for}

\begin{itemize}
\itemsep1pt\parskip0pt\parsep0pt
\item
  This is your first semester long course on quantitative methods for
  data analysis. If you have already taken a full semester statistics
  course you should not enroll.
\item
  Your are willing to spend time outside of the classroom to learn the
  course matertials, as data analysis is a skill learned by doing.
\item
  You want to be able to apply quantitative methods to your papers and
  future career.
\end{itemize}

\section{Course Objectives}\label{course-objectives}

\begin{enumerate}
\def\labelenumi{\arabic{enumi}.}
\item
  Learn the basic tools of empirical research in political science
\item
  Obtain skils in R, a highly powerful programming language used
  extenstively by academics in poitical science across the world, as
  well as the open source and data science community.
\item
  Gain real world skills that will help you obtain jobs in careers of
  the future.
\end{enumerate}

\section{Textbook}\label{textbook}

The textbook for this course will be:

Imai, Kosuke. \emph{A First Course in Quantitative Social Science}.
Princeton University Press, forthcoming.

Professor Imai has kindly made this book available to the class for
free, since it is not yet published. It is available on
\textbf{MyCourses}. Please do not share with students outside of the
class.

\section{Evaluation Policy}\label{evaluation-policy}

A description of the means of evaluation to be used in the course:

There are 1000 points available in the class. Therefore, for each 10\%
of the grade 100 points is allotted.

\begin{itemize}
\item
  \textbf{10\%}. \emph{Attendance and participation in conference
  section.} Your TA will take attendance.
\item
  \textbf{40\%}. \emph{Problem sets}. Please see the problem sets
  section. While you may talk generally with other students about these
  assignments, these problems sets must be done individually. See the
  computer code section on academic integrity. Each problem set is worth
  \textbf{100 points}. There will be four assigned problem sets.
\item
  \textbf{20\%}. \emph{In-class quizzes}. These quizzes will be closed
  book and closed notes. They will occur in class on the dates specified
  in the syllabus. Each quiz is worth \textbf{100 points}.
\item
  \textbf{30\%} \emph{Group project}. This project will consist of a
  poster and 5-10 page writeup of your findings. The group project is
  comprised of two sub-components. The individual score and the group
  score. You will receive more information about this as the course goes
  along. + Each individual in the group will be given an individual
  grade worth \textbf{100 points}, based on the evaluation of the work
  they did on the project, what they learned and their ranking of other
  group members participation. - Everyone in the group will be assigned
  a group grade woth \textbf{200 points}.

  \begin{itemize}
  \itemsep1pt\parskip0pt\parsep0pt
  \item
    \textbf{150 points} for the paper
  \item
    \textbf{50 points} for the poster
  \end{itemize}
\item
  \textbf{Extra credit-5\%}. \emph{Your participation in using the
  clicking technology.} I will use the clicker technology to ensure
  student comprehension of the material. You can receive up to
  \textbf{50 extra credit points} for using the clicking. Your points
  will be determined by the percentage of the activities you participate
  in rounded up to the nearest point.
\end{itemize}

For those who are participating for extra credit we will be using the
Turning Pointclassroom response system in class. Youwill be able to
submit answers to in-class questions using Apple or Android smartphones
andtablets, laptops, or via text message (SMS). \textbf{UNLESS OTHERWISE
INSTRUCTED THIS IS THE ONLY TIME PHONES OR COMPUTERS SHOULD BE OUT
DURING CLASS, UNLESS YOU HAVE DISCESSED WITH THE INSTRUCTOR AHEAD OF
TIME.}

Turning Point is available to all McGill students for FREE, but you have
to register. Please see XXX for the Student Quick Start Guide which
outlines how you to register, and provides a brief overview to get you
up and running on the system.

\subsection{Re-Grading}\label{re-grading}

Students who wish to contest a grade for an assignment or exam must do
so in writing (by email, sent to me) providing the reasoning behind
their challenge to the grade received, within two weeks of the day on
which the assignments are returned. The TA who graded theassignment will
re-grade your assignment, and may \textbf{raise or lower the grade}. If
you are still unsatisfied after the re-assessment, you can re-submit the
assignment to me (original copy with TA comments), along with your
justification. I will then re-evaluate the paper, but also reserve the
right to \textbf{raise or lower the grade}.

\section{Assignment Submission}\label{assignment-submission}

\begin{enumerate}
\def\labelenumi{\arabic{enumi}.}
\item
  Problem sets must be submitted in hard copy during class the day it is
  due. You must use \texttt{.Rmd} otherwise known as \texttt{rmarkdown}
  files to complete your homework. Do not submit your homework using
  Microsft Word or any other document editor. It will not be graded.
\item
  The final poster (in \texttt{PDF} format) and the final group write up
  in \texttt{PDF} format compiled from \texttt{.Rmd} must be submitted
  via MyCourses. All group members names be put on each of these
  assigmnments with a \textbf{SIGNED statement} testifying that everyone
  participated. One student for each group should be designated to turn
  in the poster and paper.
\item
  The final individual evaluation of the group project must be submitted
  individually to \textbf{MyCourses}.
\end{enumerate}

\section{Group Project}\label{group-project}

Each group should be made up of four people and cannot be more than four
people. In case of difficulty in finding the appropriate number of group
members, I have created a Google doc
(\href{https://docs.google.com/spreadsheets/d/1WbFUYr7v0FBzQlHUTkLxFZTt4A8Mib5wv5apeSnKx4M/edit?usp=sharing}{here}).
In the first tab, students can share their interest and find potential
group members. In the second tab, please register your group members and
any preliminary topic you might have.

\subsection{Interim Data Set and
Check-in}\label{interim-data-set-and-check-in}

All groups must submit a one page write up of the data set they are
going to use and the research question they are going to ask by Tuesday,
Oct 25. This should be a one page write up in \texttt{rmarkdown}
explaining the data set which you are going to use and the question you
will ask. You should also highlight your outcome variable.

\subsection{Poster}\label{poster}

Each group will present a poster the last day of class. Faculty may be
invited and you may also invite your friends. A team of faculty members
will judge the best poster. You need to print your posters. While there
is no specific size requirement, it should be poster size. I will
provide some examples. One options to print your posters is through WSR
Graphics Stewart Creagh WSR Graphics. \texttt{stewgc@videotron.ca}. You
send him what you want printed.

\subsection{Paper}\label{paper}

All groups will submit a final paper that is 5-10 pages in length. This
will be done via \textbf{MyCourses}

\subsection{E-mail and Class Mailing
List}\label{e-mail-and-class-mailing-list}

XXXX

\subsection{Make-Up Work Policy}\label{make-up-work-policy}

If you are absent for documented emergency medical or family reasons, an
alternative quiz date will be arranged. The alternative arrangement is
only open to those who can provide a valid medical/family reason for
missing the midterm or final exams. If you cannot provide a valid reason
for your absence, you will receive 0 points for the missed quiz.

Students who need to miss a class due to a religious holiday should
notify me at least fourteen days prior to the holiday. If you must miss
a class, an examination, a work assignment, or a project in order to
observe a religious holy day, you will be given an opportunity to
complete the missed work within a reasonable time after the absence.

\section{Technology Policy}\label{technology-policy}

\subsection{Screens in the Classroom
Policy}\label{screens-in-the-classroom-policy}

Mobile computer, telephone and table (MC) is strictly limited in this
class. The current
\href{http://www.psychologicalscience.org/index.php/news/releases/take-notes-by-hand-for-better-long-term-comprehension.html}{literature}
recognized that MCs inhibit learning. I have found that computers serve
to distract more students, so no computers, tablets or phones are
allowed to be used in the classroom unless specified by the instructor.
Currently, MC computing is allowed when there is an interactive quiz or
I specifically state it can be used.

The conference sections are reserved specifically for practicing coding.
Feel free to bring your own computer to these sessions.

\subsection{Recording Policy}\label{recording-policy}

\begin{itemize}
\itemsep1pt\parskip0pt\parsep0pt
\item
  No audio or video recording of any kind is allowed in class without
  the explicit permission of the instructor.
\item
  Mobile Computing devices are not to be used for voice communication
  without the explicit permission of the instructor.
\end{itemize}

\section{Academic Integrity}\label{academic-integrity}

\subsection{Course Policy on Computer
Code}\label{course-policy-on-computer-code}

Just like writing a paper, copying other people's computer code
constitutes plagiarism. Moreover data programming is learned through
trial and error. \textbf{Please do not under any circumstances copy
another students code.} If you are found to have done so, you may be
referred to the appropriate Dean. The instructors reserve the right to
use software to compare the code that has been written by different
students.

\subsection{McGill Policy}\label{mcgill-policy}

``McGill University values academic integrity. Therefore, all students
must understand the meaning and consequences of cheating, plagiarism and
other academic offences under the Code of Student Conduct and
Disciplinary Procedures'' (see \url{www.mcgill.ca/students/srr/honest/}
for more information).

\section{Other Policies}\label{other-policies}

\subsection{Language of Submission:}\label{language-of-submission}

In accord with McGill University's Charter of Students' Rights, students
in this course have the right to submit in English or in French any
written work that is to be graded.

\subsection{Disabilities Policy}\label{disabilities-policy}

As the instructor of this course I endeavor to provide an inclusive
learning environment. However, if you experience barriers to learning in
this course, do not hesitate to discuss them with me and the Office for
Students with Disabilities, 514-398-6009.

\subsection{End of Course Evaluations}\label{end-of-course-evaluations}

``End-of-course evaluations are one of the ways that McGill works
towards maintaining and improving the quality of courses and the
student's learning experience. You will be notified by e-mail when the
evaluations are available. Please note that a minimum number of
responses must be received for results to be available to students.''
\newpage

\section{Class Schedule}\label{class-schedule}

\paragraph{Week 01, 09/05 - 09/09: Introduction, Intro to
R}\label{week-01-0905---0909-introduction-intro-to-r}

\begin{itemize}
\itemsep1pt\parskip0pt\parsep0pt
\item
  READING: Ch. 1
\end{itemize}

\paragraph{Week 02, 09/12 - 09/16: Causality
1}\label{week-02-0912---0916-causality-1}

\begin{itemize}
\itemsep1pt\parskip0pt\parsep0pt
\item
  TOPIC: Randomized experiments
\item
  READING: Ch. 2.1-2.4
\item
  PROBLEM SET 1 ASSIGNED: Tuesday, Sep 13
\end{itemize}

\paragraph{Week 03, 09/19 - 09/23: Causality
2}\label{week-03-0919---0923-causality-2}

\begin{itemize}
\itemsep1pt\parskip0pt\parsep0pt
\item
  TOPIC: Observational Studies
\item
  READING: Ch. 2.5-2.7
\item
  PROBLEM SET 1 DUE: Thursday, Sep 22
\item
  PROBLEM SET 2 ASSIGNED: Thursday, Sep 22
\end{itemize}

\paragraph{Week 04, 09/26 - 09/30: Measurement
1}\label{week-04-0926---0930-measurement-1}

\begin{itemize}
\itemsep1pt\parskip0pt\parsep0pt
\item
  TOPIC: Survey sampling
\item
  READING: 3.1--3.4
\end{itemize}

\paragraph{Week 05, 10/03 - 10/07: Measurement
2}\label{week-05-1003---1007-measurement-2}

\begin{itemize}
\itemsep1pt\parskip0pt\parsep0pt
\item
  TOPIC: Clustering
\item
  READING: Ch 2.5-2.7
\item
  PROBLEM SET 2 DUE: Thursday, Oct 06
\end{itemize}

\paragraph{Week 06, 10/10 - 10/14: Prediction
1}\label{week-06-1010---1014-prediction-1}

\begin{itemize}
\itemsep1pt\parskip0pt\parsep0pt
\item
  TOPIC: Prediction and Iteration (Looping)
\item
  READING: 4.1
\item
  NO CONFERENCE THIS WEEEK
\item
  \textbf{QUIZ 1 : Tuesday, Oct 11}
\end{itemize}

\paragraph{Week 07, 10/17 - 10/21: Prediction
2}\label{week-07-1017---1021-prediction-2}

\begin{itemize}
\itemsep1pt\parskip0pt\parsep0pt
\item
  TOPIC: Regression
\item
  READING: Ch. 4.2 and 4.3
\item
  PROBLEM SET 3 ASSSIGNED: Tuesday, Oct 18
\end{itemize}

\paragraph{Week 08, 10/24 - 10/28: Probability
1}\label{week-08-1024---1028-probability-1}

\begin{itemize}
\itemsep1pt\parskip0pt\parsep0pt
\item
  TOPIC: Probability and conditional probability
\item
  READING: Ch. 6 6.1--6.3
\item
  \textbf{PAPER CHECK IN : Tuesday, Oct 25 }
\end{itemize}

\paragraph{Week 09, 10/31 - 11/04: Probability
2}\label{week-09-1031---1104-probability-2}

\begin{itemize}
\itemsep1pt\parskip0pt\parsep0pt
\item
  TOPIC: Random variables and their distributions, Large sample theorems
\item
  READING: Ch. 6.4--6.5
\item
  PROBLEM SET 3 DUE: Thursday, Nov 03
\end{itemize}

\paragraph{Week 10, 11/07 - 11/11: Uncertainty
1}\label{week-10-1107---1111-uncertainty-1}

\begin{itemize}
\itemsep1pt\parskip0pt\parsep0pt
\item
  TOPIC: Estimation
\item
  READING: Ch. 7.1
\item
  NO CONFERENCE THIS WEEEK
\item
  \textbf{QUIZ 2 : Thursday, Nov 10 }
\end{itemize}

\paragraph{Week 11, 11/14 - 11/18: Uncertainty
2}\label{week-11-1114---1118-uncertainty-2}

\begin{itemize}
\itemsep1pt\parskip0pt\parsep0pt
\item
  TOPIC: Hypothesis Tests
\item
  READING: Ch. 7.2
\item
  PROBLEM SET 4 ASSIGNED: Tuesday, Nov 15
\end{itemize}

\paragraph{Week 12, 11/21 - 11/25: Uncertainty
3}\label{week-12-1121---1125-uncertainty-3}

\begin{itemize}
\itemsep1pt\parskip0pt\parsep0pt
\item
  TOPIC: Regression with uncertainty
\item
  READING: Ch. 7.3
\end{itemize}

\paragraph{Week 13, 11/28 - 12/02: Wrap up: Poster
Presenations}\label{week-13-1128---1202-wrap-up-poster-presenations}

\begin{itemize}
\itemsep1pt\parskip0pt\parsep0pt
\item
  READING: TBD
\item
  PROBLEM SET 4 DUE: Thursday, Dec 01
\item
  \textbf{POSTER IN-CLASS: Thursday, Dec 01}
\item
  \textbf{FINAL PAPER DUE: BY TIME OF SCHEDULED EXAM}
\end{itemize}




\end{document}

\makeatletter
\def\@maketitle{%
  \newpage
%  \null
%  \vskip 2em%
%  \begin{center}%
  \let \footnote \thanks
    {\fontsize{18}{20}\selectfont\raggedright  \setlength{\parindent}{0pt} \@title \par}%
}
%\fi
\makeatother
